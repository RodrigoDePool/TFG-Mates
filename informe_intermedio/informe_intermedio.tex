\documentclass[a4paper,oneside,11pt,leqno]{article}

\usepackage[latin1]{inputenc}
\usepackage[spanish]{babel}
\usepackage{amsfonts}
\usepackage{amsmath}
\usepackage{fancyhdr}
\usepackage{epic}
\usepackage{eepic}
\usepackage{amssymb}
\usepackage{hyperref}
\usepackage{fancybox}


\textwidth = 16truecm 
\textheight = 24truecm
\oddsidemargin =-20pt
\evensidemargin = 5pt
\topmargin=-1truecm

\begin{document}
%\thispagestyle{empty}
\hrule
\vskip 6pt

\noindent{\bf TRABAJO DE FIN DE GRADO EN MATEM\'ATICAS} \\
Departamento de Matem\'aticas\\
Universidad Aut\'onoma de Madrid\\
\noindent ({\it Curso acad\'emico 2019-20}) 
\vskip 6pt \hrule

\vskip 5mm

\noindent{\bf T\'itulo del proyecto}: Teorema de clasificaci�n de superficies

\vskip 5mm

\noindent{\bf Nombre y Apellidos}: Rodrigo Alonso De Pool Alc�ntara

\vskip 5mm

\noindent{\bf Nombre del tutor(es)}: Javier Aramayona

\vskip 2cm


\centerline{\bf INFORME INTERMEDIO \footnote{ El informe debe ser elaborado por el estudiante y presentado al tutor -o tutores- que deber\'a dar su conformidad antes de ser entregado al coordinador.}}

\vskip 5mm

\begin{enumerate}

\item[1.-] {\bf Labor desarrollada hasta la fecha}: (\textit{reuniones con el tutor; b\'usqueda de bibliograf\'ia; planteamiento de los objetivos.})

- Ha habido reuniones cada dos semanas para tratar los avances y detalles de formalizaci�n de algunas demostraciones  
- Se ha utilizado el libro 'Introducci�n a la topolog�a algebra�ca' de William S. Massey.  
- Se ha utilizado el art�culo 'On the classification of noncompact surfaces' de Ian Richards.  

DESGRANAR CADA OBJETIVO(?)
Objetivos:  
- Comprender la demostraci�n del teorema de clasificaci�n de superficies compactas  
- Empezar la redacci�n del TFG   
- Aproximarnos a la demostraci�n del teorema de clasificaci�n de superficies no compactas  


\item[2.-] {\bf Esquema de los distintos apartados del trabajo}: (\textit{puede usarse como gu\'ia la propia tabla de contenidos.})

DESARROLLAR CADA SUBPUNTO HASTA EL DETALLE
\begin{enumerate}
	\item[1] Introducci�n
	\begin{enumerate}
		\item Motivaci�n
		\item Preliminares y conceptos b�sicos
		\item Resultados necesarios de topolog�a
	\end{enumerate}
	\item[2]  
	\item[3] Teorema de clasificaci�n de superficies compactas
	\item[4] Aproximaci�n al teorema de clasificaci�n de superficies no compactas
	\begin{enumerate}
		\item Nuevos conceptos necesarios
		\item Noci�n de final y frontera ideal
		\item Resultados demostrados y no demostrados :D
		\item El conjunto de Cantor en la clasifiaci�n de superficies
	\end{enumerate}
\end{enumerate}


\item[3.-] {\bf Descripci\'on del proyecto}: (\textit{motivaci\'on; principales resultados y, en su caso, aplicaciones que se esperan obtener.}) M\'aximo 2 p\'aginas.

\begin{enumerate}
	\item Motivaci�n
	\item Resultados de superficies compactas 
	\item Resultado de superficies no compactas
\end{enumerate}

\item[4.-] {\bf Grado de consecuci\'on de los objetivos y plan de trabajo a desarrollar en la segunda mitad del periodo}: 

Se ha conseguido:
\begin{enumerate}
	\item Entender la demostraci�n del teorema de clasificaci�n de superficies compactas
	\item Iniciar la redacci�n de los primeros detalles de dicha demostraci�n.
	\item Entender algunos de los nuevos conceptos necesarios para generalizar el teorema a la clasificaci�n de superficies no compactas.
\end{enumerate}
Esperado y no conseguido:
\begin{enumerate}
	\item  Finalizar la redacci�n de la demostraci�n de compactas
	\item  Entender la demostraci�n de la clasificaci�n de superficies no compactas
\end{enumerate}
Para el segundo periodo:
\begin{enumerate}
	\item Finalizar la redacci�n de la superficies compactas
	\item Entender algunos de los resultados necesarios para la demostraci�n de la clasificaci�n de superficies no compactas
	\item Entender algunas resultados interesantes de la dems de no compactas
	\item Introducir en la redacci�n la dems de no compactas
	\item Repasar todas las demostraciones utilizadas en el trabajo
	\item Preparar la presentaci�n del trabajo
	\item Hacer del trabajo una lectura lo m�s amena posible
\end{enumerate}

\item[5.-] {\bf Bibliograf\'ia usada hasta la fecha o que se piensa utilizar}: \\
Ejemplo:

\begin{thebibliography}{10}

\bibitem{Abel} 
    \textsc{Abel, N.\,H.}: 
    Beweis eines Ausdrucks, von welchem die Binomial-Formel ein einzelner Fall ist. 
    \textit{J. Reine angew. Math.} {\bf1} (1826), 159--160.

\end{thebibliography}


\end{enumerate}

\end{document}

