\documentclass[a4paper,11pt,spanish, twoside, leqno]{tfg-uam}

\usepackage[utf8]{inputenc}
\usepackage{amsfonts, amssymb, amsmath, amsthm}
\usepackage{graphicx}
\usepackage{color}

\newtheorem{teor}{Teorema}[chapter]
\newtheorem{lema}[teor]{Lema}
\newtheorem*{teorsin}{Teorema}


\theoremstyle{definition}
\newtheorem{defin}[teor]{Definici\'on}

\title{Teorema de clasificación de superficies}
\author{Rodrigo De Pool}
\tutor{Javier Aramayona}
\curso{2019-2020}


%%%%%METADATOS: rellenar la info solicitada entre llaves
\usepackage{hyperref}
\hypersetup{
	pdfinfo={
            Title={Teorema de clasificaci\'on de superficies }, %Titulo del trabajo; ejemplo: Matematicas y desarrollo
            Author={ Rodrigo De Pool}, %Autor del trabajo; ejemplo: Juan Sanchez
            Director1={javier.aramayona }, %Tutor1: en formato nombre.apellido, tal como aparece en la primera parte, antes de la arroba,  de su direcci�n de correo electr�nico de la UAM; ejemplo: fernando.soria
            Director2={ }, %Tutor2: en formato nombre.apellido, tal como aparece en la primera parte, antes de la arroba,  de su direcci�n de correo electr�nico de la UAM
            Ndirectores={1 }, %Numero total de directores: 1 � 2
            Tipo={TFG}, %no tocar
            Curso={2018-19}, %no tocar
            Palabrasclave={ },% Palabras clave del trabajo, separadas por comas y sin acentos ni espacios; ejemplo: morfismos, formas modulares, ecuaciones elipticas
				}
}
%%%%%%%%%%%%%%%%%%%%%%%%%%%%%%%

\begin{document}




\begin{abstract}[spanish]
Aquí va un resumen.
\end{abstract}
\begin{abstract}[english]
	Here goes the abstract
\end{abstract}


\mainmatter


\chapter{Introducci\'on y preliminares}\label{chap1}
\setcounter{page}{1}

El trabajo presenta un estudio de las superficies topológicas y su clasificación. Primero, se estudiarán algunas herramientas, como el de suma conexa o triangulación, que serán útiles para demostrar el teorema de clasificación bajo la hipótesis de compacidad. Luego, nos acercaremos al resultado de [KEREKJARTOS] en el que se retira la exigencia de compacidad, para ello tendremos que introducir nuevas nociones como el de finales o frontera ideal. Por último, examinaremos la demostración constructiva de Ian Richards \cite{ian} que permite construir una superficie representante para cada clase de equivalencia topológica, y, además, desvela una interesante relación entre las 2-variedades conexas no compactas y los subconjuntos del conjunto de Cantor.

\section{Introducci\'on a las superficies topológicas}

Para formalizar el concepto de superficie necesitamos primero definir las variedades topológicas:
\begin{defin}
	Un conjunto, $\mathcal{X}$, dotado de una topología, diremos que es una \textit{n-variedad} si es Hausdorff y para todo punto existe un entorno homeomorfo a una bola abierta n dimensional.
\end{defin}

Llamaremos entonces \textit{superficie} a toda 2-variedad conexa que cumple el segundo axioma de numerabilidad. La definición generaliza el concepto intuitivo que se suele manejar de superficies. Algunos ejemplos de superficies son: La esfera con la topología de usual o un toro con la topología de inducida de $\mathcal{R}^3$, que son ambas orientables;o una banda de M\"{o}bius con la topología de subespacio, para el caso de una superficie no orientable.

[IMAGEN 1]

\subsection{Superficies compactas}

El toro y el plano proyectivo son dos superficies esenciales para entender la clasificación de superficies compactas. Por ende, se plantean algunos ejemplos para familiarizarnos con estas variedades.

Tomando el cuadrado cerrado $ X = \{ (x,y) \in R^2: -1\leq x\leq 1,\quad -1\leq y \leq 1  \} $ con la topología de subespacio, el \textit{toro}  se construye de identificar:
\begin{align*}
(x,-1)\equiv(x,1) \quad x\in [-1,1]\\
(-1,y)\equiv(1,y) \quad y\in [-1,1] 
\end{align*}
Y dotar al conjunto resultante con la topología de espacio cociente. En la [REF A IMG2] se representa gráficamente el toro, las aristas con la misma letra se identifican siguiendo el sentido indicado por la flecha.

[IMAGEN 2]

Es un ejercico simple pero extenso  demostrar que  el toro así definido es homeomorfo a la estructura de donut que se suele estudiar.

Por su parte, el \textit{plano proyectivo} parte del mismo conjunto $X$ y se construye con las identificaciones:
\begin{align*}
(x,-1)\equiv(-x,1) \quad x\in [-1,1]\\
(-1,y)\equiv(1,-y) \quad y\in [-1,1] 
\end{align*}
Utilizando en este caso también la topología cociente. Equivalentemente se puede definir como el espacio cociente que resulta de identificar los puntos diametralmente opuestos de $S^2$.

[IMAGEN 3]

Las gráficas [REF IMG1] y [REF IMG2] nos permiten establecer una notación algebraica para referirnos a estas superficies: 
Inciando en una arista cualquiera recorremos la figura en el sentido de las agujas del reloj y anotamos cada letra que encontramos, agregando un exponente a la -1 en caso de que el sentido de su flecha sea inverso al del recorrido. Con esto describiríamos [REG IMG1] como $aba^{-1}b^{-1}$ y [REF IMG2] como $abab$. La notación nos permite referirnos a nuevas superficies fácilmente, por ejemplo, $aba^{-1}b$ se correspondería con [REF IMG4] y $aa^{-1}$ con [REF IMG5]. 

[IMAGEN 4](Botella de klein, OYE INTENTA HACER LA DEMS PARA ESTA SUPERFICIE)

[IMAGEN 5]

El siguiente lema [REF A LEMA] nos permitirá comprobar que todas las superficies tratadas hasta ahora son compactas.

\begin{lema}
	Sea $X$ un espacio topológico e $Y$  un espacio cociente que resulta de identificar puntos en $X$. Entonces:
	\begin{align*}
	\text{$X$ compacto}\Rightarrow\text{$Y$ compacto}
	\end{align*}
\end{lema}
\begin{proof}
	Sea $f:X\longrightarrow Y$ la función cociente que asocia a cada punto su clase de equivalencia; $f$ es claramente sobreyectiva y, por ser cociente, es continua.\\
	Siendo $X$ compacto tenemos entonces que $f(X)=Y$ también lo es.
\end{proof}



Definimos a continuación un operador que nos permitirá generar infinidad de nuevas superficies partiendo de las ya estudiadas (de hecho, si se me permite el \textit{spoiler}, podremos generar todas las superficies compactas posibles).


\begin{defin}
	Sean $S_1$ y $S_2$ dos superficies disjuntas. Llamaremos \textit{suma conexa} de ambas, $S_1\#S_2$, a la superficie que resulta de:TODO CONTINUAR
\end{defin}

[Nota: sigo acaso demasiado la estructura que hay en el libro de massey? es esto un plagio? No sabría plantearlo de otra forma, su hilo argumental tiene todo el sentido del mundo]




\chapter{Algunos resultados de topología}
Cosas que empecé a escribir pero que probablemente no haga falta incluir en el TFG. Son definiciones elementales de topo y algunos lemas

\begin{defin}
	Un conjunto $\mathcal{X}$ es no conexo si existen dos  conjuntos cerrados, $\mathcal{X}_1$ y $\mathcal{X}_2$, tal que $\mathcal{X}=\mathcal{X}_1\cup\mathcal{X}_2$ y $\mathcal{X}_1\cap\mathcal{X}_2=\emptyset$.
\end{defin}
\begin{defin}
	Un conjunto $\mathcal{X}$ se dice conexo si no cumple la definición anterior.
\end{defin}
\begin{defin}
	Sea $\mathcal{X}$ un espacio topológico con topología $\mathcal{T}_X$, sea $\mathcal{Y}$ un conjunto, y $f$ una función  $f:\mathcal{X}\longrightarrow\mathcal{Y}$. Entonces definimos la topología cociente:
	\begin{align*}
	\mathcal{T}_Y = \{U\subset\mathcal{Y}: f^{-1}(U)\in\mathcal{T}_X\}
	\end{align*} 
	\begin{itemize}
		\item 
		Se puede comprobar $T_Y$ genera en efecto una topología de $Y$
		\item
		$T_Y$ es la topología más fina que hace continua a $f$
		\item 
		Es usual trabajar con $Y$ como una partición o conjunto de clases de equivalencia de $X$
	\end{itemize}
\end{defin}

\begin{lema}\label{conexoAconexo}
	Sean $X$ e $Y$ espacios topológicos y $f: X \longrightarrow Y$ continua, entonces: 
	\begin{align*}
	X \textrm{ conexo} \Rightarrow Y \textrm{conexo}
	\end{align*}
\end{lema}
\begin{proof}
	content...
\end{proof}


\begin{lema}\label{compactoAcompacto}
	Sean $X$ e $Y$ espacios topológicos y $f: X \longrightarrow Y$ continua, entonces: 
	\begin{align*}
	X \textrm{ compacto} \Rightarrow f(X) \textrm{ compacto}
	\end{align*}
\end{lema} (FALTA COROLARIO DE CUANDO F ES EXHAUSTIVA)

\begin{lema}\label{XcompactoYt2fcontinua}
	Sean $X$ e  $Y$ espacios topológicos, $X$ un compacto, $Y$ un espacio de Haussdorf y $f: X \longleftrightarrow Y$ continua, entonces $f$ es cerrada.
\end{lema} (FALTA COROLARIO  DE QUE Y TIENE LA TOPO COCIENTE DETERMINADA POR F)

\begin{defin}
	Una n-variedad es un espacio topológico de Haussdorff tal que todo punto tiene un entorno abierto homeomorfo a la bola abierta n-dimensional.
\end{defin}

\begin{defin}
	A una 2-variedad conexa la llamaremos superficie.
\end{defin}

\begin{defin}
	Una triangulación de una superficie compacta, $S$, consiste en subconjuntos cerrados, ${T_1, ..., T_n}$, que cubren a $S$ y una familia de homeomorfismos, ${\phi_1, ..., \phi_n}$, que cumplen:
	\begin{align*}
	\phi_i: T'_i \longrightarrow T_i
	\end{align*}
	Donde $T'_i$ es un triángulo del plano $\mathbb{R}^2$. Además, tomando $T_i$ y $T_j$ con $i\neq j$, se cumple una de las siguientes condiciones:
	\begin{itemize}
		\item 
		Son conjuntos totalmente disjuntos
		\item 
		Comparten un solo vértice en común (Llamamos vértice a todo elemento de $S$ que se corresponde por algún $\phi_i$ con un vértice en el plano).
		\item 
		Tienen toda una arista en común y solo eso (Llamamos arista a la preimagen de una arista de alguún $T'_i$ por $\phi_i$).
	\end{itemize}
	
\end{defin}

\begin{teor}{Teorema de Tibor Radó}\label{teoremaDeTriangulacion}
	Toda $S$ superficie compacta es triangulable.
\end{teor}(NO LO DEMOSTRAREMOS)

LEMAS DE TRIANGULACIÓN
\begin{lema}\label{lema1detriangulacion}
	Sea $S$ una superficie triangulable entonces una arista lo es de exactamente dos triángulos.
\end{lema}(TODO: FALTA DEMS)

\begin{lema}\label{lema2detriangulacion}
	Sea  $S$ una superficie triangulable y $v\in S$ un vértice en esa triangulación, entonces podemos ordenar el conjunto de todos los triángulos con vértice $v$ cíclicamente,  $T_0, T_1, ..., T_n = T_0$, de manera que $T_i$ y $T_{i+1}$ tienen toda una arista en común para todo $0\leq i\leq n-1$.
\end{lema}(TODO: FALTA DEMS)


\begin{thebibliography}{10}

%% TODO: VER COMO SE TIENEN QUE PONER LAS CITAS
\bibitem{massey} 
    \textsc{William S. Massey}: 
    Introducción a la topología algebraica. 
    \textit{Editorial Reverté} {\bf1} (2006), 1--29.

\bibitem{ian}
    \textsc{Ian Richards}
    \textit{Classification of non compact surfaces...o algo asi}
    
    
\end{thebibliography}
\cleardoublepage


\end{document}
